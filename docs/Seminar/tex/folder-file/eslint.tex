ESLint یک Linter برای زبان برنامه‌نویسی Javascript هست که با استفاده از Node.js به وجود اومده است.

همونطور که میدونین Javascript مانند زبانهای دیگه همچون Java و ... نیست و کامپایلری ندارد و کدها مستقیما در مرورگر اجرا میشن. در زبانهای دیگه که کامپایلر وجود دارد، در زمان compile کردن کد، اگر مشکلی وجود داشته باشد در اکثر موارد بیان میشه و compile با موفقیت به پایان نمیرسه ولی در Javascript به دلیل عدم وجود compiler، مشکلات خودشون رو در زمان اجرا شدن کد در مرورگر نمایش میدن.

ESLint این ارورها رو برای شما پیدا میکنه و جلوی چنین اتفاقاتی رو میگیره. شما بیشتر به دنبال چه نوع ارورهایی هستید که در کدهاتون رخ نده؟
\begin{itemize}
	\item 
	جلوگیری از حلقه‌های بی‌نهایت یا infinite loop با قرار دادن شرط نامناسب
	\item 
	اطمینان از اینکه همه متدهای getter، چیزی رو return میکنند.
	\item 
	جلوگیری از قرار دادن console.log در کدها
	\item 
	چک کردن case‌های تکراری در switch
	\item 
	چک کردن کدهای غیر قابل دسترس. مثلا بعد از return هر کدی رو قرار بدیم، unreachable یا غیر قابل دسترسی میشه.
\end{itemize}

ESLint بسیار منعطف و با قابلیت تنظیم بالا هست. شما میتونین مشخص کنید که چه rule هایی باید برای کدهای شما چک بشه. همچنین میتونین مشخص کنید که چه نوع استایل استانداردی رو میخواید برای کدهاتون قرار بدین. خیلی از rule‌ها بصورت پیش فرض غیر فعال یا فعال هستند و شما میتونین با استفاده از فایل .eslintrc برای کل پروژه‌ها یا یک پروژه خاص، تنظیمات مورد نظرتون رو قرار بدین.