\chapter{کلیات طرح، اهداف، انگیزه}
\addcontentsline{toc}{section}{مقدمه}
\section*{مقدمه}
 برای جمع اوری اطلاعاتی در مورد سیستم رزرو هتل به چندین هتل مراجعه نمودم و توانستم اطلاعاتی در مورد سیستم رزرو هتل از طریق صحبت کرن با کارکنان محل و همچنین دیدن امکانات و برنامه های نصب شده و در حال اجرا بدست بیاورم.همچنین با مراجعه به وب سایت هتل توانستم چند رزرو اینترنتی نیز را انجام دهم و بعد با همکاری مسئولان هتل چگونگی پذیرش مسافر را از نزدیک و به صورت عینی مشاهده نمودم. همچنین  به جز این موارد توانستم درهنگام مراجعه چند مسافر برای پذیرش مراحل پذیرش مسافر را از نزدیک مشاهده کنم و در مجموع اطلاعات بسیار مهمی در مورد مسئله مورد تحقیق بدست آورم.ازجمله فیلدهایی که در مورد مسافران باید داشته باشیم تا بتوانیم یک بانک اطلاعاتی مناسب طراحی کنم.
از موارد دیگر که باید گفته شود این است که با دیگر قسمتهای برنامه نصب شده درهتل نیز آشنا شدم و در واقع امکانات جانبی نرم افزار را نیز مشاهده نمودم.
هچمچنین توانستم نشانی شرکت ایجاد کننده نرم افزار را پیدا کنم و با مساعدت یکی از مسئولین آن شرکت توانستم با مسئول طراحی بانک صحبت کنم و اطلاعات خوبی بدست آورم. در مجموع با اطلاعات جمع آوری توانستم تا حدودی بتوانم مسئله را بیشتر مورد ارزیابی قراردهم.سپس با توجه به اطلاعات بدست آمده   طراحی ساختار نرم افزار را شروع کردم و بعد از طراحی شروع به برنامه نویسی نمودم.

\section{تعریف و بیان کار}
دراین پروژه موارد زیر از دانشجو طلبیده می شد:
\begin{enumerate}
	\item
سیستم توانایی رزرو اینترنتی و پذیرش مسافر را دارا باشد.
\item
این سیستم دارای انواع گزارشها از مسافران و کارکنان باشد.
\item
سیستم توانایی جستجوهای بر اساسهای مختلف از مسافران  را داراباشد.
\end{enumerate}

\section{سابقه و ضرورت}
\section{هدف‌ها}
\section{کاربردها}
\section{مراحل کار}
\section{روش کار}
\section{ساختار گزارش}
