\documentclass[20pt,a5paper]{report}
\usepackage{graphicx,geometry}
\usepackage{multirow,hhline}
\usepackage[hidelinks]{hyperref}

\usepackage{xepersian}

\settextfont{Adobe Arabic}
\geometry{a5paper,top = 0.5cm, left = 1cm, right = 1cm, bottom = 0.5cm}


\begin{document}
\noindent \textbf{مورد استفاده:}
ویرایش پروژه
\\
\textbf{شرح مختصر :UC}
در این قسمت کارفرما پروژه خود را ویرایش میکند.
\\
\textbf{پيش شرط:}
ورود به داشبورد کارفرما.
\\
\textbf{سناريو اصلی:}
\begin{enumerate}
\item
شروع
\item
کارفرما دکمه ویرایش پروژه را انتخاب میکند و سیستم فرم اطلاعات پروژه را به کارفرما نمایش میدهد.
\item
کارفرما فرم را اصلاح میکند و با دکمه ارسال، فرم اصلاح شده را به سیستم ارسال میکند.
\item
سیستم اطلاعات فرم را بررسی میکند و اطلاعات را در بانک اطلاعات بروزرسانی میکند.
\item
پایان
\end{enumerate}
\textbf{پس شرط:}
ندارد.
\\
\textbf{سناريوهای فرعی:}
\\ \\
\textbf{سناريو فرعی 1:}
خطا در اطلاعات فرم ویرایش پروژه
\\
\textbf{شرح مختصر :UC}
این سناریو در مرحله ۴ سناریو اصلی در صورت خطا در اطلاعات فرم اجرا میشود.
\begin{enumerate}
\item
شروع
\item
اطلاعات فرم بررسی میشود و خطاها مشخص میشوند.
\item
یک پیغام به کارفرما نمایش داده میشود و درخواست اصلاح اطلاعات فرم را دارد.
\item
از مرحله 3 سناریو اصلی ادامه پیدا میکند.
\item
پایان
\end{enumerate}
\textbf{سناريو فرعی 2:}
پروژه با موفقیت ویرایش شود
\\
\textbf{شرح مختصر :UC}
این سناریو در مرحله ۴ سناریو اصلی در صورت موفقیت آمیز بودن ویرایش پروژه اجرا میشود.
\begin{enumerate}
\item
شروع
\item
اطلاعات فرم بررسی میشود و یک پیغام به کارفرما نمایش داده میشود که اطلاعات با موفقیت ثبت شده است.
\item
از مرحله 4 سناریو اصلی ادامه پیدا میکند.
\item
پایان
\end{enumerate}

\textbf{پس شرط:}
ندارد .


\centering
\vfill
\lr{\LaTeX}
\end{document}
